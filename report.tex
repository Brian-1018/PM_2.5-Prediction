\documentclass[conference]{IEEEtran}
\IEEEoverridecommandlockouts
% The preceding line is only needed to identify funding in the first footnote. If that is unneeded, please comment it out.
\usepackage{cite}
\usepackage{amsmath,amssymb,amsfonts}
\usepackage{algorithmic}
\usepackage{graphicx}
\usepackage{textcomp}
\usepackage{xcolor}
\def\BibTeX{{\rm B\kern-.05em{\sc i\kern-.025em b}\kern-.08em
    T\kern-.1667em\lower.7ex\hbox{E}\kern-.125emX}}
\begin{document}

\title{Include your project title here\\

}

\author{\IEEEauthorblockN{1\textsuperscript{st} Given Name Surname}
\IEEEauthorblockA{\textit{dept. name of organization (of Aff.)} \\
\textit{name of organization (of Aff.)}\\
City, Country \\
email address or ORCID}
\and
\IEEEauthorblockN{2\textsuperscript{nd} Given Name Surname}
\IEEEauthorblockA{\textit{dept. name of organization (of Aff.)} \\
\textit{name of organization (of Aff.)}\\
City, Country \\
email address or ORCID}
\and
\IEEEauthorblockN{3\textsuperscript{rd} Given Name Surname}
\IEEEauthorblockA{\textit{dept. name of organization (of Aff.)} \\
\textit{name of organization (of Aff.)}\\
City, Country \\
email address or ORCID}
}

\maketitle

\section{Minimum Requirements of Project}
Please refer to the final project guidelines document available on Blackboard Learn. The document lists the minimum requirements, grading rubric, and a tentative project timeline. While working on the project plan, you can also review the lecture slides on machine learning system design and applications.

The minimum required page limit of the final project report is 3.

\section{Abstract}
Explain the problem that you are trying to explore and your findings. This is similar to an abstract in a research paper.


The following subsections are required (you can include more sections as applicable)-
\subsection{Introduction}
Introduce the problem statement. This section is similar to the introduction sections in the research papers.
\subsection{Related Work}
Include related works in the research field. Also, mention how your work is novel or different from the existing works. Include citations as they appear in a technical research paper.
\subsection{Data Collection}
Describe the dataset and features that you used.
\subsection{Data Visualization}
Include figures and explain the observations. The main goal is to understand the data/features.
\subsection{Paper Used (optional, if using an existing paper)}
\subsection{Approach}
\subsection{Methodology}
Methodology — Include the technical details of the machine learning algorithm and any analysis scenarios (if applicable). Analysis scenarios should explain how you structured your analysis and considered different subsets or conditions within your domain.
For example, in an ADHD monitoring study, you might compare classification results using only daytime data versus a full 24-hour window. Similarly, if your dataset spans multiple countries or genres, you could design experiments that analyze each subset individually.

\subsection{Results}
Include tables/figures to list the results and explain them in detail. Based on these results, this section should describe the insights/intuition you observe.
\section{Individual Team Member's Responsibilities}
Include details on individual members' responsibilities and roles.

\section{Optional-Above and Beyond}
Please include the work you considered for the bonus section and its results.

\section{Conclusion and Future Work}


\section{References}








\end{document}
